\section{Conclusiones.}

El desarrollo de este artículo ha permitido experimentar todas las etapas del desarrollo de modelos analíticos poniendo en práctica todo el contenido teórico estudiado durante la asignatura. Ha resultado un trabajo muy interesante y una gran aportación para reforzar los conceptos estudiados. 

La selección de variables realizada aparentemente no ha sido demasiado efectiva dado que todos los algoritmos se mueven en unos rangos de eficacia entre el 55\% y el 65\%, ligeramente más eficaces que el modelo aleatorio. La lista de algoritmos en función de los resultados obtenidos ha sido:

\begin{itemize}

\item{ID3.}
\item{J48.}
\item{Naive Bayes}
\item{Conjunto de algoritmos: Logistic, TAN, JRIP.}
\item{Conjunto de algoritmos: IB1, IB5, IB10.}

\end{itemize}

Ha podido influir también el sobremuestreo realizado. Quizás con otra proporción de casos positivos y negativos se hubiera podido mejorar los resultados.

También se ha puesto de manifiesto que la herramienta Weka es un conjunto de algoritmos potentes de DataMining pero dista mucho de ser una herramienta útil y utilizable en un entorno productivo empresarial. Los problemas para manejar grandes cantidades de información y los estrictos formatos de los conjuntos de datos que necesita para realizar las validaciones de los diferentes modelos la convierten en una herramienta bastante improductiva que genera grandes demoras y pérdidas de tiempo en el desarrollo de modelos analíticos.

Para ello existen otras herramientas en el mercado para un uso intensivo profesional, como SAS Enterprise Miner e IBM SPSS Modeler que probablemente solucionen parte de las dificultades encontradas con el software WEKA con la contrapartida del alto coste económico de las licencias necesarias.