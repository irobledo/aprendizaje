\section{Introducción.}

La Asociación de Veteranos Paralíticos de EEUU \cite{PVA} es una organización sin ánimo de lucro cuyo objetivo principal es mejorar la vida de soldados heridos gravemente durante el desarrollo de su profesión. Según ellos mismos describen en su sitio web oficial en internet, buscan proporcionar los medios y alternativas suficientes para que estas personas puedan desarrollar su libertad e independencia dentro de las limitaciones impuestas por su estado físico.

Como organización sin ánimo de lucro, todos los fondos que utiliza en sus diferentes programas, son recaudados a partir de las aportaciones realizadas por los ciudadanos de los EEUU. La asociación cuenta con una importante base de datos de donantes que han realizado sus aportaciones en el pasado. Realiza una intensa actividad de comunicación sobre los mismos para mantenerlos informados de los nuevos programas, de las nuevas actividades y de este modo incentivar la aportación de nuevas donaciones.

Todas estas actividades de comunicación son caras y el presupuesto que tiene la asociación es bastante reducido. Por ello ha decidido que quiere invertir en los mecanismos necesarios para realizar una óptima y eficiente gestión de ese presupuesto, maximizando los resultados obtenidos del mismo.

A lo largo de este documento se detalla el desarrollo de algunas herramientas de inteligencia de negocio que permitirán a la asociación seleccionar y priorizar los donantes presentes en la base que tienen mayor potencial para realizar una nueva donación.

Este ejercicio fue propuesto en el ámbito del concurso KDD CUP \cite{KDD-CUP} organizado por el grupo de DataMining de la organización ACM \cite{SIGKDD-ACM} en 1998. Todos los conjuntos de datos necesarios están a libre disposición en la dirección web del KDD Contest \cite{KDD-CUP-1998}.