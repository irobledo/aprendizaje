\section{Descripci�n del problema.}

La PVA (Asociaci�n de Veteranos Paral�ticos de Am�rica) realizo un mailing a 3,5 millones de anteriores donantes en Junio de 1997. La PVA esta muy interesada en construir un modelo anal�tico de selecci�n de donantes para recibir la campa�a espec�fico para los donantes inactivos cuya �ltima donaci�n ocurr�o hace m�s de 12 y menos de 24 meses. Estos inactivos son 191.779 donantes que han sido repartidos en dos conjuntos de datos diferentes:

\begin{itemize}

\item{Training data set: conjunto de datos de aprendizaje formado por 95.412 donantes de los cuales 4.843 hicieron una nueva donaci�n tras recibir el mailing (5,1 \%).}

\item{Validation data set: conjunto de datos de validaci�n formado por 96.367 donantes de los cuales 4.873 hicieron una nueva donaci�n tras recibir el mailing (5,1 \%).}

\end{itemize}

El conjunto de entrenamiento y el conjunto de validaci�n han sido construidos de forma aleatoria a partir de la poblaci�n inicial con lo que en principio, no debe existir ning�n sesgo en los mismos, permitiendo realizar una correcta validaci�n de la eficiencia de los modelos una vez que sean construidos.

Cada mailing enviado por la asociaci�n le supuso un coste de 0,68\$ por pieza producida y enviada. 

\subsection{Descripci�n del contenido de los conjuntos de datos.}

Cada uno de los conjuntos de datos anteriormente comentados contiene los siguientes bloques de informaci�n:

\begin{itemize}

\item{Informaci�n detallada del donante: fecha de primera donaci�n, estado de residencia, origen de los datos del donante, etc.}
\item{Informaci�n de las campa�as realizadas sobre cada donante por la PVA en los �ltimos 24 meses y donaciones recibidas en el mismo periodo.}
\item{Variables resumen del total de donaciones realizadas por cada donante y estructuradas por tipo de campa�a a la que fueron atribuidas.}
\item{Variables socidodemogr�ficas obtenidas del censo de los EEUU asociadas a cada donante a partir de la direcci�n de residencia geolocalizada.}

\end{itemize}

El total de variables incluidas en los bloques anteriores son 481. Se distribuyen por cada uno de los bloques del siguiente modo:

\begin{itemize}

\item{Informaci�n detallada del donante: 94 variables.}
\item{Informaci�n de las campa�as realizadas por la PVA en los �ltimos 24 meses y donaciones: 91 variables}
\item{Variables resumen del total de donaciones realizadas por cada cliente: 10 variables.}
\item{Variables socidodemogr�ficas obtenidas del censo de los EEUU: 286 variables.}

\end{itemize}

\subsection{Modelos objetivo.}

El objetivo final de estos trabajo es proporcionar a la PVA una o varias herramientas que le permitan seleccionar dentro de los 191.779 donantes inactivos, aquellos m�s propensos a realizar la donaci�n de forma que puedan reducir el coste del env�o de las campa�as, y no tener que enviar a los 191.779. Las cifras medias actuales de una campa�a podr�an ser las siguientes:

\begin{itemize}

\item{191.000 env�os a 0,69\$ por envio supone un coste de campa�a de 131.790,00\$.}
\item{Una respuesta en torno al 5% suponen aproximadamente 9.550 donaciones.}
\item{Una donaci�n media en torno a los 79\$ supone recibir aproximadamente 769.000\$ en donaciones.}
\item{Esto supone un margen de unos 637.000\$ para la campa�a.}

\end{itemize}

Para cubrir este objetivo se van a desarrollar diferentes clasificadores con diferentes t�cnica de aprendizaje que permitan clasificar a los potenciales donantes en funci�n de si donan o no donan. Se seleccionar� el clasificador con mejores resultados para aplicarlo a futuras campa�as.